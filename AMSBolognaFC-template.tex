%%%%%%%%%%%%%%%%%%%%%%%%%%%%%%%%%%%%%%%%%%%%%%%%%%%%%%%%%%%%%%%%%%%%%%%%%%%%%%%%
% AMS Beamer series / Bologna FC / Template
% Andrea Omicini
% Alma Mater Studiorum - Università di Bologna
% mailto:andrea.omicini@unibo.it
%%%%%%%%%%%%%%%%%%%%%%%%%%%%%%%%%%%%%%%%%%%%%%%%%%%%%%%%%%%%%%%%%%%%%%%%%%%%%%%%
\documentclass[presentation,apice]{beamer}\mode<presentation>{\usetheme{AMSBolognaFC}}
%\documentclass[presentation]{beamer}\mode<presentation>{\usetheme{AMSBolognaFC}}
%\documentclass[handout]{beamer}\mode<handout>{\usetheme{AMSBolognaFC}}
%%%%%%%%%%%%%%%%%%%%%%%%%%%%%%%%%%%%%%%%%%%%%%%%%%%%%%%%%%%%%%%%%%%%%%%%%%%%%%%%
\usepackage{wasysym}
\usepackage[yyyymmdd]{datetime}
\renewcommand{\dateseparator}{}
\def\BibTeX{{\rm B\kern-.05em{\sc i\kern-.025em b}\kern-.08em
    T\kern-.1667em\lower.7ex\hbox{E}\kern-.125emX}}
\renewcommand{\thefootnote}{\fnsymbol{footnote}}
% version
\newcommand{\templatemajor}{1}
\newcommand{\templateminor}{6}
\newcommand{\templatepatch}{20230214}
\newcommand{\templateversion}{\templatemajor.\templateminor.\templatepatch}
%%%%%%%%%%%%%%%%%%%%%%%%%%%%%%%%%%%%%%%%%%%%%%%%%%%%%%%%%%%%%%%%%%%%%%%%%%%%%%%%
\title[AMS Bologna FC Beamer Style]
{AMS Bologna FC Beamer Style}
%
\subtitle[AMS Series Templates]
{AMS Series Templates}
%
\author[\sspeaker{Omicini} \and Else]
{\speaker{Andrea Omicini} \and Nobody Else\\\href{mailto:andrea.omicini@unibo.it}{andrea.omicini@unibo.it} \and \href{mailto:nobody.else@unibo.it}{nobody.else@unibo.it}}
%
\institute[DISI, Univ.\ Bologna]
{Dipartimento di Informatica -- Scienza e Ingegneria (DISI)\\\textsc{Alma Mater Studiorum} -- Universit{\`a} di Bologna}
%
\date[v.\ \templateversion]{version \templateversion}
%
%%%%%%%%%%%%%%%%%%%%%%%%%%%%%%%%%%%%%%%%%%%%%%%%%%%%%%%%%%%%%%%%%%%%%%%%%%%%%%%%
\begin{document}
%%%%%%%%%%%%%%%%%%%%%%%%%%%%%%%%%%%%%%%%%%%%%%%%%%%%%%%%%%%%%%%%%%%%%%%%%%%%%%%%

%/////////
\frame{\titlepage}
%/////////

%%===============================================================================
%\section*{Outline}
%%===============================================================================
%
%%/////////
%\frame[c]{\tableofcontents[hideallsubsections]}
%%/////////

%===============================================================================
\section{Premises}
%===============================================================================

%/////////
\begin{frame}[c]{Academic Presentations}
%
\begin{itemize}
	\item technical presentations are common in the academic environment
	\item \alert{\LaTeX{}}\footnote{\uurl{http://latex.org}} is typically used for scientific literature
	\begin{itemize}
		\item for better formulas, bibliography, typography, \ldots
	\end{itemize}
	\item so, \LaTeX{} is the most typical \emph{toolbox} for scientific presentations
	\begin{itemize}
		\item which easily translates into teaching presentations, too
	\end{itemize}
\end{itemize}
%
\end{frame}
%/////////

%/////////
\begin{frame}[c]{Beamer}
%
\begin{itemize}
	\item \alert{Beamer} is a \LaTeX{} document class for creating presentation \emph{slides}, with a wide range of \emph{templates} and a set of features for making \emph{slideshow effects}
	\begin{description}
		\item[repo] \uurl{https://github.com/josephwright/beamer}
		\item[CTAN] \uurl{https://ctan.org/pkg/beamer}
	\end{description}
	\item even though other styles for \LaTeX{} presentations exist, Beamer is by far the best way today
\end{itemize}
%
\end{frame}
%/////////

%/////////
\begin{frame}[c]{AMS}
%
\begin{description}
	\item[AMS] Alma Mater Studiorum--Università di Bologna
	\item[Cesena] Campus of Cesena
	\begin{itemize}
		\item hence the AMS Cesena Campus logo on the bottom-left corner
	\end{itemize}
\end{description}
%
\end{frame}
%/////////

%/////////
\begin{frame}[c]{Aim}
%
\begin{itemize}
	\item many occasions for presentation
	\item many different people in the group
	\item a common reference could be useful
	\item hence, this style
	\item based on random Beamer styles, and using some references for the colours
\end{itemize}
%
\end{frame}
%/////////

%/////////
\begin{frame}[c]{Who}
%
\begin{itemize}
	\item mostly, myself
	\item and, collaborators of mine---the \alert{good} ones \textcolor{bolognafcred}{\smiley{}}
	\item everybody else, \emph{be my guest!}
\end{itemize}
%
\end{frame}
%/////////

%===============================================================================
\section{Use}
%===============================================================================

%/////////
\begin{frame}[c,allowframebreaks]{Files}
%
\begin{block}{Style files}
The files
\begin{itemize}
	\item \texttt{beamercolorthemebolognafc.sty}
	\item \texttt{beamerthemeAMSBolognaFC.sty}
	\item \texttt{almacesena-background.pdf}
\end{itemize}
should be placed either in the local folder with the main \texttt{.tex} file, or, in your Beamer system directory, e.g.
\begin{itemize}
	\item \texttt{/Users/\{username\}/Library/texmf/tex/latex/local/beamer/}
\end{itemize}
\end{block}
%
\begin{block}{BST files}
The files
\begin{itemize}
	\item \texttt{apalike-AMS.bst}
\end{itemize}
should be placed either in the local folder with the main \texttt{.tex} file, or, in your Beamer system directory, e.g.
\begin{itemize}
	\item \texttt{/Users/\{username\}/Library/texmf/bibtex/bst/local/}
\end{itemize}
\end{block}
%
\end{frame}
%/////////

%/////////
\begin{frame}[c,fragile]{Declaration}
%
\begin{block}{\texttt{\textbackslash{}documentclass}}
Your main Beamer \texttt{.tex} file should open with the declaration
%
\begin{verbatim}
    \documentclass[presentation]{beamer}
        \mode<presentation>{\usetheme{AMSBolognaFC}}
\end{verbatim}
%
so as to use the AMS Bologna FC Beamer style 
\end{block}
%
\end{frame}
%/////////

%/////////
\begin{frame}[c,fragile]{Bibliography Style}
%
\begin{block}{\texttt{apalike-AMS}}
Your main Beamer \texttt{.tex} file should include the declaration
\begin{verbatim}
    \bibliographystyle{apalike-AMS}
\end{verbatim}	
so as to use the AMS Bologna FC \BibTeX{} style 
\end{block}
%
\end{frame}
%/////////

%/////////
\begin{frame}[c,fragile]{Template}
%
\begin{block}{\texttt{AMSBolognaFC-template.tex}}
This template's sources can be used as a simple example of how yo use this Beamer style
\end{block}
%
\end{frame}
%/////////


%===============================================================================
\section{Style}
%===============================================================================

%/////////
\begin{frame}[c,allowframebreaks]{Colours for AMS Bologna FC}
%
\begin{exampleblock}{\texttt{\textcolor{bolognafcblue}{bolognafcblue}}}
	\begin{description}
		\item[HEX] \texttt{\#1A2F48}
		\item[RGB] \texttt{26,47,72}
	\end{description}
\end{exampleblock}
\begin{exampleblock}{\texttt{\textcolor{bolognafcred}{bolognafcred}}}
	\begin{description}
		\item[HEX] \texttt{\#A21C26}
		\item[RGB] \texttt{162,28,38}
	\end{description}
\end{exampleblock}
\framebreak
\begin{block}{\texttt{\textcolor{bolognafcwhite}{bolognafcwhite}}}
	\begin{description}
		\item[HEX] \texttt{\#FFFFFF}
		\item[RGB] \texttt{255,255,255}
	\end{description}
\end{block}
\begin{alertblock}{\texttt{\textcolor{bolognafcsilver}{bolognafcsilver}}}
	\begin{description}
		\item[HEX] \texttt{\#ECECEC}
		\item[RGB] \texttt{236,236,236}
	\end{description}
\end{alertblock}
%
\end{frame}
%/////////

%/////////
\begin{frame}[c,fragile]{Blocks}
%
\begin{block}{This is a \texttt{block} environment}
\begin{verbatim}
\begin{block}
...
\end{block}
\end{verbatim}
\end{block}
%
\begin{exampleblock}{This is an \texttt{exampleblock} environment}
\begin{verbatim}
\begin{exampleblock}
...
\end{exampleblock}
\end{verbatim}
\end{exampleblock}
%
\begin{alertblock}{This is an \texttt{alertblock} environment}
\begin{verbatim}
\begin{alertblock}
...
\end{alertblock}
\end{verbatim}
\end{alertblock}
%
\end{frame}
%/////////

%===============================================================================
\section{New Commands}
%===============================================================================

%/////////
\begin{frame}[c,allowframebreaks,fragile]{Citations}
%
\begin{alertblock}{\texttt{\textbackslash{}ccite} command---e.g., \ccite{bibtex-patashnik88}}
\begin{verbatim}
\ccite{bibtex-patashnik88}
\end{verbatim}
\begin{itemize}
	\item to be used instead of standard \texttt{\textbackslash{}cite} command
	\item prints as \ccite{bibtex-patashnik88}
	\item can be used as a note\ccite{bibtex-patashnik88}, with no space before
\end{itemize}
\end{alertblock}
%
\begin{exampleblock}{\texttt{\textbackslash{}cccite} command---e.g., \cccite{bibtex-patashnik88}}
\begin{verbatim}
\cccite{bibtex-patashnik88}
\end{verbatim}
\begin{itemize}
	\item a lighter version of the \texttt{\textbackslash{}ccite} command over non-dark, non-light backgrounds
	\begin{itemize}
		\item as here above in \texttt{examplebox} header
	\end{itemize}
	\item can be used as a note with no space before, in the same way as \texttt{\textbackslash{}ccite}
\end{itemize}
\end{exampleblock}
%
\end{frame}
%/////////

%/////////
\begin{frame}[c,fragile]{URLs}
%
\begin{block}{\texttt{\textbackslash{}uurl} command}
\begin{itemize}
	\item to be used instead of standard \texttt{\textbackslash{}url} command
	\item[e.g.] \verb|\uurl{http://apice.unibo.it}| prints as \uurl{http://apice.unibo.it}
\end{itemize}
\end{block}
%
\begin{block}{\texttt{\textbackslash{}uuurl} command---e.g., \uuurl{http://apice.unibo.it}}
\begin{itemize}
	\item to be used instead of standard \texttt{\textbackslash{}url} command over dark backgrounds
	\item[e.g.] see \verb|\uuurl{http://apice.unibo.it}| above in this \texttt{block} header
\end{itemize}
\end{block}
%
\end{frame}
%/////////

%/////////
\begin{frame}[c,fragile]{Alert}
%
\begin{block}{\texttt{\textbackslash{}aalert} command---e.g., \aalert{alerted text}}
\begin{itemize}
	\item to be used instead of standard \texttt{\textbackslash{}alert} command over dark backgrounds
	\item[e.g.] see \verb|\aalert{alerted text}| above in this \texttt{block} header
\end{itemize}
\end{block}
%
\end{frame}
%/////////

%/////////
\begin{frame}[c,fragile,allowframebreaks]{Speaker(s) \emph{vs.} Authors}
%
\begin{exampleblock}{\texttt{\textbackslash{}speaker} command--e.g., \speaker{Diego Zorro}}
\begin{itemize}
	\item to be used within \texttt{\textbackslash{}author} standard Beamer command to single out the actual speaker among the authors
	\item[e.g.] as in
\begin{verbatim}
	\author[Garcia \and Zorro]
	{Sarg Garcia \and \speaker{Diego Zorro}}
\end{verbatim}
	\item and in the author specification of this template
\end{itemize}
\end{exampleblock}
%
\begin{block}{\texttt{\textbackslash{}sspeaker} command--e.g., \sspeaker{Diego Zorro}}
\begin{itemize}
	\item to be used within \texttt{\textbackslash{}author} standard Beamer command to single out the actual speaker among the authors in the short form
	\item[e.g.] as in
\begin{verbatim}
	\author[Garcia \and \sspeaker{Zorro}]
	{Sarg Garcia \and \speaker{Diego Zorro}}
\end{verbatim}
	\item and in the author specification of this template
\end{itemize}
\end{block}

\end{frame}
%/////////

%===============================================================================
\section{New Options}
%===============================================================================

%/////////
\begin{frame}[c,fragile,allowframebreaks]{\texttt{apice} option}
%
\begin{alertblock}{\texttt{apice} Beamer option}
\begin{itemize}
	\item the option \texttt{apice} can be used in the presentation preamble
	\begin{itemize}
		\item as in
		\begin{verbatim}
\documentclass[presentation,apice]{beamer}...
		\end{verbatim}
	\end{itemize}
	\item enables the display of the \BibTeX{} field \texttt{apice}
	\begin{itemize}
		\item containing the relative URL of the publication on the \alert{\href{http://apice.unibo.it/}{\textsf{APICe}}} Wiki
	\end{itemize}
	\item properly linked to the corresponding \href{http://apice.unibo.it/}{\textsf{APICe}} page
\end{itemize}
\end{alertblock}
%
\begin{exampleblock}{Example}
%
The following \BibTeX{} entry in the \texttt{.bib} file with the \texttt{apice} option enabled
%
{\tiny\begin{verbatim}
@manual{bibtex-patashnik88,
    apice = {BibtexPatashnik88},
    author = {Patashnik, Oren},
    booktitle = {\BibTeXing},
    month = {8~} # feb,
    organization = {CTAN, The Comprehensive TeX Archive Network},
    title = {\BibTeX{}ing},
    url = {http://mirrors.ctan.org/biblio/bibtex/base/btxdoc.pdf},
    year = 1988}
\end{verbatim}}
%
would translate into the following entry in the presentation bibliography
%
\begin{thebibliography}{}
\bibitem[Patashnik, 1988]{bibtex-patashnik88}
Patashnik, O. (1988).
\\[-0.1em] {\em \BibTeX{}ing}.
\\[-0.0em] CTAN, The Comprehensive TeX Archive Network
\\[-0.4em] \apicepar{BibtexPatashnik88} \uuurl{http://mirrors.ctan.org/biblio/bibtex/base/btxdoc.pdf}.
\end{thebibliography}
%
\end{exampleblock}
%
\end{frame}
%/////////

%===============================================================================
\section*{}
%===============================================================================

%/////////
\frame{\titlepage}
%/////////

%===============================================================================
\section*{\refname}
%===============================================================================

%%%%
\setbeamertemplate{page number in head/foot}{}
%/////////
\begin{frame}[c,noframenumbering]{\refname}
%\begin{frame}[t,allowframebreaks,noframenumbering]{\refname}
%	\tiny
	\scriptsize
%	\footnotesize
	\bibliographystyle{apalike-AMS}
	\bibliography{AMSBolognaFC-template}
\end{frame}
%/////////

%%%%%%%%%%%%%%%%%%%%%%%%%%%%%%%%%%%%%%%%%%%%%%%%%%%%%%%%%%%%%%%%%%%%%%%%%%%%%%%%
\end{document}
%%%%%%%%%%%%%%%%%%%%%%%%%%%%%%%%%%%%%%%%%%%%%%%%%%%%%%%%%%%%%%%%%%%%%%%%%%%%%%%%
