%%%%%%%%%%%%%%%%%%%%%%%%%%%%%%%%%%%%%%%%%%%%%%%%%%%%%%%%%%%%%%%%%%%%%%%%%%%%%%%%
% SD Lab -- About the course
% Giovanni Ciatto
% Alma Mater Studiorum - Università di Bologna
% mailto:giovanni.ciatto@unibo.it
%%%%%%%%%%%%%%%%%%%%%%%%%%%%%%%%%%%%%%%%%%%%%%%%%%%%%%%%%%%%%%%%%%%%%%%%%%%%%%%%
%\documentclass[handout]{beamer}\mode<handout>{\usetheme{default}}
%
\documentclass[presentation]{beamer}\mode<presentation>{\usetheme{AMSBolognaFC}}
%\documentclass[handout]{beamer}\mode<handout>{\usetheme{AMSBolognaFC}}
%%%%%%%%%%%%%%%%%%%%%%%%%%%%%%%%%%%%%%%%%%%%%%%%%%%%%%%%%%%%%%%%%%%%%%%%%%%%%%%%
\usepackage{sd-lab-common}
\usepackage{sd-lab-guidelines}
%%%%%%%%%%%%%%%%%%%%%%%%%%%%%%%%%%%%%%%%%%%%%%%%%%%%%%%%%%%%%%%%%%%%%%%%%%%%%%%%
\title[\currentLab{} -- Guidelines]{
	Guidelines for project and exam
}
%
\subtitle{\courseName{} (\courseAcronym) / Module \moduleN{}}
%
\author[\sspeaker{\gcShort} \& \mmShort]{
	\speaker{\mmFull} \and \gcFull
	\\
	\mmEmail \and \gcEmail
}
%
\institute[\disiShort, \uniboShort]{\disi{} (\disiShort)\\\unibo}
%
\date[A.Y. \academicYear{}]{Academic Year \academicYear{}}
%
\begin{document}

\maketitle

\begin{frame}[c]\frametitle{Outline}
    % \begin{multicols}{2}
        \tableofcontents[sectionstyle=show/show, subsectionstyle=show/show, subsubsectionstyle=show/show]
    % \end{multicols}
\end{frame}

\section{Follow the rules}
\subsection{Mandatory Labs}

\begin{frame}[c,allowframebreaks]{About the labs}

    \begin{block}{All you need to know}
        \begin{itemize}
            %
            \item there will be only 3-4 mandatory labs
            %
            \item you can do them whenever you want
            \\
            $\rightarrow$ \alert{before taking appointment for the lab discussion}
            %
            \item we suggest to do them during the end of the lab class or in a few days
            %
        \end{itemize}
    \end{block}

    \framebreak

    \begin{block}{How we evaluate}
        \begin{itemize}
            %
            \item no need to worry
            %
            \item you can look at the result of the CI
            %
            \begin{itemize}
                %
                \item {\color{green} green light} $\rightarrow$ you could have done all right
                %
                \item {\color{red} red light} $\rightarrow$ you have done something wrong
                %
            \end{itemize}
            %
            \item again, don't worry if you have a red light
            %
            \begin{itemize}
                %
                \item look at the logs and try to figure out what went wrong
                %
                \item look at the tests and at the CI file
                %
                \item we do not care too much if you can explain your solution
                %
            \end{itemize}
            %
        \end{itemize}
        %
    \end{block}
    %
    We can have a look together at the \bluehref{\gitlabGroup/lab-1}{first lab}.
\end{frame}

\subsection{Project workflow}

\begin{frame}[c]{Project rules}
    %
    \bluehref{\projectRules}{Here} you can find the exhaustive rules for this year.
    
    \begin{block}{In a nutshell}
    %
    \begin{enumerate}
        %
        \item Reserving / proposing a project
        %
        \begin{enumerate}
            %
            \item post your request on the \bluehref{\projectForum}{project forum}
            %
            \item\label{item:wait} wait for our reply (usually it is a go)
            %
            \begin{enumerate}
                %
                \item go $\rightarrow$ write the initial report
                %
                \item no go $\rightarrow$ fine-tune the proposal (back to 1.2)
                %
            \end{enumerate}
            %
        \end{enumerate}
        %
        \item Beginning the project
        %
        \begin{itemize}
            %
            \item we will give you a repository
            \\
            \alert{we will look only at that repo!}
            %
            \item we suggest to include \alert{gitignore}, \alert{gitattributes} and a readme
            %
        \end{itemize}
        %
        \item Submit the project
        %
        \begin{itemize}
            %
            \item follow the \bluehref{\template}{template} for the final report
            % 
        \end{itemize}
    \end{enumerate}
    %
    \end{block}
\end{frame}

\section{Follow the guidelines}
\subsection{Technical aspects}

\begin{frame}[c,allowframebreaks]{Structure of the Lab}

    An incremental path providing a taste of all the aforementioned issues: %(showing how main notions are built \& exploited C\&DS)
    %
    \medskip
    %
    \begin{enumerate}
        \item we will first present the fundamentals of
        %
        \begin{itemize}
            \item \alert{build automation} and \emph{dependency management} with Gradle
            \item \alert{containerisation} and \alert{orchestration} with Docker
        \end{itemize}

        \medskip

        \item we will then recall the basics of
        %
        \begin{itemize}
            \item \alert{multi-threading} and \alert{asynchronous programming}, on the JVM
            \item (byte-)stream-oriented \alert{communication via Sockets}, on the JVM
        \end{itemize}

        \medskip

        \item we will then present the notion of \alert{(de-)serialization}
        %
        \begin{itemize}
            \item and its usage for (un)marshalling data in remote communications
            \item proposing practical exercises with the Jackson data-processing library
        \end{itemize}

        \framebreak

        \item we will then present the notion of \alert{ReST API}
        %
        \begin{itemize}
            \item showing how they can be \alert{formalised} via Swagger
            \item discussing how \alert{web servers} can be realised on the JVM, via Javalin
            \item discussing how \alert{web clients} can be realised on the JVM
        \end{itemize}

        \medskip

        \item we will then describe state-of-the-art tools for
        %
        \begin{itemize}
            \item building \alert{service-oriented applications}, e.g. via gRPC and Protobuff
            \item \alert{temporally uncoupling} interacting entities, e.g. via RabbitMQ
            \item \alert{replicating} data over a number of distributed machines, e.g. via etcd
        \end{itemize}

        \medskip

        \item finally, we will exemplify notable \alert{agent-oriented infrastructures}
        %
        \begin{itemize}
            \item such as \jade{}\ccite{jadebook-2007}, TuCSoN\ccite{tucson-jir98}, or TuSoW\ccite{tusow-icccn2019}
        \end{itemize}

    \end{enumerate}

\end{frame}

\subsection{Behavioural aspects}

\begin{frame}[c,allowframebreaks]{Required Technologies and Skills}

    \begin{block}{Legend}
        \begin{multicols}{2}
            \begin{itemize}
                \item[$\checkmark$] we assume you know this topic
                \item[$\rightarrow$] we teach this topic
            \end{itemize}
        \end{multicols}
    \end{block}

    \framebreak

    \begin{alertblock}{Required}
        \begin{itemize}
            \item[$\checkmark$] basic understanding of computer networks
            %
            \begin{itemize}
                \item ISO/OSI and TCP/IP stacks
            \end{itemize}

            \item[$\checkmark$] distributed version control systems (DVCS) \& \alert{Git}
            %
            \begin{itemize}
                \item useful resources: \cite{pianiniDvcs, proGit}
            \end{itemize}

            \vfill

            \item[$\rightarrow$] build automation tools and \alert{Gradle}
            %
            \begin{itemize}
                \item useful resources: \cite{pianiniBuildAutomation, gradleUserGuide}
            \end{itemize}

            \vfill

            \item[$\rightarrow$] containerisation, orchestration and \alert{Docker}

            \vfill

            \item OO programming in Java, and, in particular:
            %
            \begin{itemize}
                \item[$\checkmark$] input/output API (useful resources: \cite{ProgrammizJavaIO})
                \item[$\checkmark$] collections API (useful resources: \cite{Naftalin2006, Bloch2008, JavaCollectionsCheatsheets})
                \item[$\checkmark$] streams and lambdas API (useful resources: \cite{Warburton2014, Bloch2008})
                \item[$\checkmark$] multi-threading API (useful resources: \cite{Lea1999, Oaks2004, Garg2004, Goetz2006})
                \item[$\rightarrow$] asynchronous programming API
            \end{itemize}

        \end{itemize}
    \end{alertblock}

    \begin{exampleblock}{Useful}
        \begin{itemize}
            \item[$\checkmark$] shell scripting and \alert{Bash}
            \item[$\checkmark$] basic IDE configuration and usage (\alert{Eclipse} or \alert{IntelliJ Idea})
        \end{itemize}
    \end{exampleblock}

    \begin{alertblock}{Setting up your own environment}
        \begin{itemize}
            \item[!] follow the instructions provided here: \cite{envSetup}
        \end{itemize}
    \end{alertblock}

\end{frame}



\begin{frame}[c]{About Projects}
    \begin{itemize}
        \item detailed rules here
        \\
        \uurl{\projectRules}

        \vfill

        \item workflow overview
        %
        \begin{enumerate}
            \item \alert{choose} a project or \alert{propose} one
            \item reserve your project on the \bluehref{https://virtuale.unibo.it/mod/forum/view.php?id=611834}{Projects forum}
            \item submit an \alert{initial report}, describing your own requirements
            \item receive a Git repository for tracking the development of your artefacts
            \item develop your projects
            \item write the \alert{final report}
            \item submit your project \alert{code \& report} and ask for lab activity check
            \item set up an appointment for discussing your project
        \end{enumerate}

        \vfill

        \item group projects are allowed (max 4 persons)
        %
        \begin{itemize}
            \item rule of thumb: $\sim90$ working hours per person per project
        \end{itemize}
    \end{itemize}
\end{frame}

\begin{frame}[c, allowframebreaks]{Sorts of Projects}

    \begin{block}{Classic DS Project}
        Develop a distributed application
        %
        \begin{itemize}
            \item[eg] web application, distributed video game, etc
            \item Workflow:
            %
            \begin{enumerate}
                \item Sketch the idea
                \item Design
                \item Write tests
                \item Implement
                \item Pack / deploy for reusability
            \end{enumerate}
            \item Previous projects in this category: \cite{Sd2021ProjectGuessR,Sd2021ProjectCAHu}
            \item[!] Innovation is limited here, thus we expect
            %
            \begin{itemize}
                \item design to be optimal, implementation to be complete
                \item all aspects of software engineering
                %
                \begin{itemize}
                    \item[ie] testing, packaging, continuos integration
                \end{itemize}
            \end{itemize}
        \end{itemize}
    \end{block}

    \begin{block}{Advanced DS Project}
        Provide an implementation for some distributed algorithm / protocol / architecture / model from the literature
        %
        \begin{itemize}
            \item[eg] a consensus protocol, a distributed store, a data sharing protocol, etc.
            \item Workflow:
            %
            \begin{enumerate}
                \item Select \& study a paper form the literature
                \item Design your implementation for the proposed contribution
                \item Prototype an implementation
                \item Pack / deploy for a demo
            \end{enumerate}
            \item Previous projects in this category: \cite{Sd2021ProjectZyzzyva, Sd2021HybridQuorum}
            \item[!] Innovation is high here, thus we
            %
            \begin{itemize}
                \item mostly care about the proposed design
                \item implementation can be sub-optimal or embryonic
                \item other aspects of software engineering have lower priority
            \end{itemize}
        \end{itemize}
    \end{block}

    \begin{block}{Research DS Project}
        Help us extending some research product of ours with some feature
        %
        \begin{itemize}
            \item[eg] TuCSoN\ccite{tusow-icccn2019}, TuSoW\ccite{tusow-icccn2019}, LPaaS\ccite{lpaas-bdcc2}, tuProlog\ccite{2pkt-softwarex}, etc.
            \item Workflow:
            %
            \begin{enumerate}
                \item Study the target research product
                \item Collect requirements by interacting with us
                \item Co-design the feature by interacting with us
                \item Implement \& test the feature
            \end{enumerate}
            \item Previous projects in this category: \cite{Sd2021ProjectTusowCSharpInterface}
            \item[!] Innovation is high here, but the activity is very constrained
            %
            \begin{itemize}
                \item design and implementation must be expert-level
                \item testing, and continuos integration are mandatory
                \item you will be considered part of development team
                \item you will be part of the author list of any subsequent paper
            \end{itemize}
        \end{itemize}
    \end{block}

    \begin{block}{Technology Know-How DS Project}
        Study a technology of choice in depth, producing a technical report
        %
        \begin{itemize}
            \item Workflow:
            %
            \begin{enumerate}
                \item Select a technology, study its documentation
                \item Understand its major abstraction
                \item Experiment with its deployment, and ordinary usage
                \item Stress it with extra-ordinary usage
                \item Report about the technology and the experiment
            \end{enumerate}
            \item Previous projects in this category: \cite{Sd2021ProjectEtcd,Sd2021ProjectCEPH,Sd2021ProjectKubernetes}
            \item[!] Focus here is on the technical report, other than the experiments
            %
            \begin{itemize}
                \item software artefacts here consist of the code of the experiments
                \item avoid copy-pasting the documentation, use your own words
                \item experiment should cover more than the official ones
            \end{itemize}
        \end{itemize}
    \end{block}
\end{frame}

%===============================================================================
\section*{}
%===============================================================================
\frame{\titlepage}

%===============================================================================
\section*{\bibname}
%===============================================================================

\setbeamertemplate{page number in head/foot}{}
%\\\\\\\\\\\\\\\\\\\\\
\begin{frame}[t,allowframebreaks,noframenumbering]\frametitle{\refname}
    % \begin{frame}[c]\frametitle{\refname}
    %    \footnotesize
    %    \scriptsize
        \tiny
    \bibliographystyle{plain}
    \bibliography{sd-lab-about}
\end{frame}
%\\\\\\\\\\\\\\\\\\\\\

%%%%%%%%%%%%%%%%%%%%%%%%%%%%%%%%%%%%%%%%%%%%%%%%%%%%%%%%%%%%%%%%%%%%%%%%%%%%%%%
\end{document}
%%%%%%%%%%%%%%%%%%%%%%%%%%%%%%%%%%%%%%%%%%%%%%%%%%%%%%%%%%%%%%%%%%%%%%%%%%%%%%%%
