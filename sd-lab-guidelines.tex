%%%%%%%%%%%%%%%%%%%%%%%%%%%%%%%%%%%%%%%%%%%%%%%%%%%%%%%%%%%%%%%%%%%%%%%%%%%%%%%%
% SD Lab -- About the course
% Giovanni Ciatto
% Alma Mater Studiorum - Università di Bologna
% mailto:giovanni.ciatto@unibo.it
%%%%%%%%%%%%%%%%%%%%%%%%%%%%%%%%%%%%%%%%%%%%%%%%%%%%%%%%%%%%%%%%%%%%%%%%%%%%%%%%
%\documentclass[handout]{beamer}\mode<handout>{\usetheme{default}}
%
\documentclass[presentation]{beamer}\mode<presentation>{\usetheme{AMSBolognaFC}}
%\documentclass[handout]{beamer}\mode<handout>{\usetheme{AMSBolognaFC}}
%%%%%%%%%%%%%%%%%%%%%%%%%%%%%%%%%%%%%%%%%%%%%%%%%%%%%%%%%%%%%%%%%%%%%%%%%%%%%%%%
\usepackage{sd-lab-common}
\usepackage{sd-lab-guidelines}
%%%%%%%%%%%%%%%%%%%%%%%%%%%%%%%%%%%%%%%%%%%%%%%%%%%%%%%%%%%%%%%%%%%%%%%%%%%%%%%%
\title[\currentLab{} -- Guidelines]{
	Guidelines for project and exam
}
%
\subtitle{\courseName{} (\courseAcronym) / Module \moduleN{}}
%
\author[\sspeaker{\gcShort} \& \mmShort]{
	\speaker{\mmFull} \and \gcFull
	\\
	\mmEmail \and \gcEmail
}
%
\institute[\disiShort, \uniboShort]{\disi{} (\disiShort)\\\unibo}
%
\date[A.Y. \academicYear{}]{Academic Year \academicYear{}}
%
\begin{document}

\maketitle

\begin{frame}[c]\frametitle{Outline}
    % \begin{multicols}{2}
        \tableofcontents[sectionstyle=show/show, subsectionstyle=show/show, subsubsectionstyle=show/show]
    % \end{multicols}
\end{frame}

\section{Follow the rules}
\subsection{Mandatory Labs}

\begin{frame}[c,allowframebreaks]{About the labs}

    \begin{block}{All you need to know}
        \begin{itemize}
            %
            \item there will be only 3-4 mandatory labs
            %
            \item you can do them whenever you want
            \\
            $\rightarrow$ \alert{before taking appointment for the lab discussion}
            %
            \item we suggest to do them during the end of the lab class or in a few days
            %
        \end{itemize}
    \end{block}

    \framebreak

    \begin{block}{How we evaluate}
        \begin{itemize}
            %
            \item no need to worry
            %
            \item you can look at the result of the CI
            %
            \begin{itemize}
                %
                \item {\color{green} green light} $\rightarrow$ you could have done all right
                %
                \item {\color{red} red light} $\rightarrow$ you have done something wrong
                %
            \end{itemize}
            %
            \item again, don't worry if you have a red light
            %
            \begin{itemize}
                %
                \item look at the logs and try to figure out what went wrong
                %
                \item look at the tests and at the CI file
                %
                \item we do not care too much if you can explain your solution
                %
            \end{itemize}
            %
        \end{itemize}
        %
    \end{block}
    %
    We can have a look together at the \bluehref{\gitlabGroup/lab-1}{first lab}.
\end{frame}

\subsection{Project workflow}

\begin{frame}[c]{Project rules}
    %
    \bluehref{\projectRules}{Here} you can find the exhaustive rules for this year.
    
    \begin{block}{In a nutshell}
    %
    \begin{enumerate}
        %
        \item Reserving / proposing a project
        %
        \begin{enumerate}
            %
            \item post your request on the \bluehref{\projectForum}{project forum}
            %
            \item\label{item:wait} wait for our reply (usually it is a go)
            %
            \begin{enumerate}
                %
                \item go $\rightarrow$ write the initial report
                %
                \item no go $\rightarrow$ fine-tune the proposal (back to 1.2)
                %
            \end{enumerate}
            %
        \end{enumerate}
        %
        \item Beginning the project
        %
        \begin{itemize}
            %
            \item we will give you a repository
            \\
            \alert{we will look only at that repo!}
            %
            \item we suggest to include \alert{gitignore}, \alert{gitattributes} and a readme
            %
        \end{itemize}
        %
        \item Submit the project
        %
        \begin{itemize}
            %
            \item follow the \bluehref{\template}{template} for the final report
            % 
        \end{itemize}
    \end{enumerate}
    %
    \end{block}
\end{frame}

\section{Follow the guidelines}
\subsection{Technical aspects}

\begin{frame}[c,allowframebreaks]{Design Design Design}
    
    {\centering
    \includegraphics[width=\textwidth]{figures/boromir}}
    
    \framebreak

    The most common mistake of your previous colleagues was starting their project by writing code.
    %
    \alert{This is wrong!}
   
     \begin{block}{What you should think about before writing code}
        \begin{itemize}
            %
            \item entities involved
            %
            \item functionalities
            %
            \item communication
            %
            \item \alert{faults}
            %
            \begin{itemize}
                %
                \item detection
                %
                \item countermeasures
                %
            \end{itemize}
            %
            \item technologies
            %
            \item etc...
            %
        \end{itemize}
    \end{block}
    
    \framebreak
    
    %
    \vfill
    %
    \begin{block}{You must rely on diagrams in your design}
        %
        \begin{itemize}
            %
            \item sequence diagram
            %
            \item message communication diagram
            %
            \item state diagram
            %
            \item class diagram
            %
            \item use case diagram
            %
        \end{itemize}
        %
    \end{block}
    %
    A free online tool for drawing UML diagrams is \bluehref{https://plantuml.com/}{plantUML}.
    %
    %This is an example of a hello world \bluehref{https://www.plantuml.com/plantuml/uml/SoWkIImgAStDuNBCoKnELT2rKt3AJx9IS2mjoKZDAybCJYp9pCzJ24ejB4qjBk42oYde0jM05MDHLLoGdrUSoeLkM5u-K5sHGY9sGo6ARNHr2QY66kwGcfS2SZ00}{sequence diagram} that you can edit at will to make practice.

\end{frame}
    
\begin{frame}[c,allowframebreaks]{Use the tools properly}
    
    \begin{figure}
        \centering
        \animategraphics[loop,width=0.5\textwidth]{10}{figures/sequence_rocket_launcher/rocket_launcher-}{0}{73}
    \end{figure}
    
    \framebreak
    
    \begin{block}{The curse of UML diagrams}
        %
        For some unknown reason students struggle with UML diagrams.
        %
        \begin{itemize}
            %
            \item missing diagrams
            %
            \item diagram are present but they are not commented
            \\
            $\rightarrow$ \alert{sometimes not even referenced!}
            %
            \item wrong diagrams
            \\
            $\rightarrow$ wrong use of the elements of the diagrams, different from the idea you had in mind, missing elements
            %
        \end{itemize}
        %
    \end{block}
    
    \framebreak
    
    \begin{block}{Museum of horrors}
        %
        \begin{itemize}
            %
            \item state diagram:
            %
            \begin{itemize}
                %
                \item \bluehref{https://www.plantuml.com/plantuml/uml/SoWkIImgAStDuOhMYbNGrRLJSCxFoqjDBealoOztSU92uIc0f8jI4qjIunFpubFpKWhoC\_DAk325m0RvP1QNfEPbvgLpmLceubOA6ObvAJaWyQBKmjBKuf9YBWUW4K3N0000}{wrong} example and a possible \bluehref{https://www.plantuml.com/plantuml/uml/SoWkIImgAStDuOhMYbNGrRLJSCxFoqjDBealoOztSU92uIc0f8jI4qjIunFpubFpKWhoC_DAk325m0RvP1QNfEPbvgLpmLceubOA6ObvAJaWyQBKmjBKuf9YBWUW4K3N0000}{solution}
                %
            \end{itemize} 
            %
            \item sequence diagram:
            %
            \begin{itemize}
                %
                \item \bluehref{https://www.plantuml.com/plantuml/uml/RP11QiCm44NtEiMiilC5GYcXP5iez01XAlscehBaZaRkzIF7IHX3rlF_-uyvL6NHDYRdZiLh4Hg6L_g4p90zQHn1oby9L97WDPRirkCrzrxokHpVzkH_u7zfHxKJ-0nJTBVCXYkeKyaLbhNERpRR55ZleHEz30Dzi4PaNH1_3LN9Sq8EB1GmiUZOZ_NUi8wFXXtOBIOzsw5TvudgoRXQw4EZp_kMhdtY76YE-AVoJJnpJElZKkb6jy8MlPpCOvzQUm4hp6kDiNC7vD36T040}{wrong} example 1 and possible \bluehref{https://www.plantuml.com/plantuml/uml/ZP1DIWGn48NtEKMjPjSN64G6cQqWU88oUOx1TAgjghJNb_v1qo3eEkzzzULTp5czZIM4gqGNfM6ufsVe33sJvn7a_590XeBeMMfO7xVSezIRN1_E5DyGt-b4vg9yGrSCU-R0LTWarT2pjlE7csrAxD-Xks6W9_QCwWlXey6S8XVKn4e94wkFg6w_6kw57dzQFA0pBz-o_uh-E4GkiPMrJLmJpHcV9A_4C-UoZLR4BUo50tXv-MjJ1JkwKJlR42wegOtb2m00}{solution}
                %
                \item \bluehref{https://www.plantuml.com/plantuml/uml/ROunJiOm38Ltdy8Z3Fy5650t3e1w02jvfH5E2xO3zFOa0sf1B8dazpr_dZp4ebKbcXt6jziZFyC-O0ziU6_5m2NpiXjl-EGNgQ2H_q7JYlGwqa4JYGR3RBUxtXKrI3uONyLB-3rT5yp-Ou97_vVco4HGno_kIIgwnY21yv8nlSIICx-f732EdMF5XJfbtbA2-z9KZr71loN2becc_hQgtm00}{wrong} example 2 and possible \bluehref{https://www.plantuml.com/plantuml/uml/ROv1JiDG34JtFiKieL8lq0NgZWCW3c3bJsY4au3j3-JsvEjBG54ttioyUJv7gcUTJUGvw3WU7_14_w2Vu7olanDJmRWw5dxYHMTApw-gmuN3QqjRMqGjGVVLxtPMpQbb0x-cc3BkvUz0atOJ8-4SQt0Ve0Mnh7qaheZ79ZaSi4kTclpzgoK_bHEq5BoPRdJaYhpmNwkhemu38yPgjZM4Do7XKaJE-wspVGC0}{solution}
            \end{itemize}
            %
        \end{itemize}
        %
    \end{block}
    %
\end{frame}

\begin{frame}[c,allowframebreaks]{Everything according to plan}
    %
    \begin{block}{We do not live in a simple world}
        %
        \begin{itemize}
            %
            \item distributed systems are \alert{distributed}
            \\
            $\rightarrow$ you will appreciate the consequences of that as the course goes on
            %
            \item things \alert{never} go as planned
            %
            \begin{itemize}
                %
                \item what if an entity of your system suddenly disconnects?
                %
                \item can I detect that?
                %
                \item should I do something if this happens?
                %
                \item what if there are Byzantines?
                %
            \end{itemize}
            %
        \end{itemize}
        %
    \end{block}
     
    \framebreak
    
    \begin{block}{Stem the tide}
        %
        \begin{itemize}
            %
            \item think about the previous questions when you are proposing the project
            %
            \item address them when designing the system
            %
            \item test the unwanted scenarios while implementing
            %
            \item \bluehref{https://people.mpi-sws.org/~rupak/Papers/SoftwareModelChecking.pdf}{software model checking}
            %
            \item \bluehref{https://en.wikipedia.org/wiki/Petri_net}{Petri nets}
        \end{itemize}
        %
    \end{block}
    
    \framebreak
    
    \begin{block}{What happens if you just ignore all of this?}
        %
        \begin{itemize}
            %
            \item not (properly) working system
            %
            \item you loose time and energy jury rigging
            %
            \item we may ask you to resubmit your project
            %
            \item bad practice for a software engineer
            %
        \end{itemize}
        %
    \end{block}
    %
\end{frame}

\begin{frame}[c,allowframebreaks]{Test and Continuos Integration}
    
    \begin{block}{Do not do something a machine can do (better)}
        %
        \begin{itemize}
            %
            \item \alert{testing is paramount!}
            %
            \item you can do it manually or automatically
            %
            \begin{itemize}
                \item manually $\rightarrow$ you run the system and perform some actions, then you judge if the system behaved accordingly to what you expected.
                %
                Manual test could be used to test the graphical interface.
                %
                \item automatically $\rightarrow$ you write some code that uses functions, entities or the entire system and run it.
                %
                You can use unit tests for functions and classes, integration test for entities (in particular communication between them).
                %
            \end{itemize}
            %
            \item we encourage to test automatically and programmatically (i.e., using the continuous integration = CI) as much as possible
            %
        \end{itemize}
        %
    \end{block}
    
    \framebreak
    
    \begin{block}{What and how to test in the right way}
        %
        \begin{itemize}
            %
            \item unit test
            %
            \begin{itemize}
                %
                \item single functions or classes
                %
                \item by using values you expect the test to pass
                %
                \item by using values you expect the test to \alert{fail!}
                %
            \end{itemize}
            %
            \item integration test
            %
            \begin{itemize}
                %
                \item entities and communication between them
                %
                \item you could consider to use docker
                %
                \item disconnections to simulate faults
                %
                \item wait to simulate slow network
                %
            \end{itemize}
            %
            \item End to end test
            %
            \begin{itemize}
                %
                \item the whole system
                %
                \item you should use docker
                %
                \item write logs of what each component is doing and for events
                %
                \\
                %
                $\rightarrow$ then, you analyse the logs!
            \end{itemize}
            %
        \end{itemize}
        %
    \end{block}
    
\end{frame}


\subsection{Behavioural aspects}

\begin{frame}[c,allowframebreaks]{Time management}
    
    \begin{figure}
        \centering
        \includegraphics[height=0.8\textheight]{figures/waiting-meme}    
    \end{figure}
    
    
    \framebreak
    
    \begin{block}{Waiting for Godot}
        %
        Students tend to wait indefinitely before submitting a proposal
        %
        \begin{itemize}
            %
            \item they do not know what (project) to do
            %
            \item \alert{they do not feel ready yet!}
            %
            \item all the rest of the universe of reasons
            %
        \end{itemize}
        %
    \end{block}
    %
    Spoiler: you will never feel ready enough (for this particular project as for mostly any non trivial choices / commitments in your life).
    %
    So, just submit a proposal eventually.
    %
    
\end{frame}
    
    
\begin{frame}[c,allowframebreaks]{Being smart}
    %  
    \begin{block}{Exploit your colleagues' previous experience}
        %
        % Meme to insert
        %
        \begin{itemize}
            %
            \item have a look at the projects of the previous years
            %
            \item \bluehref{https://apice.unibo.it/xwiki/bin/view/Courses/Series/Ds/Projects}{here} you can see all the ongoing and completed projects
            %
            \begin{itemize}
                %
                \item you can read some of them to be inspired
                %
                \item you can also subscribe to old Virtuale DS courses and read the posts in the projects forum
                %
            \end{itemize}
            %
        \end{itemize}
        %
    \end{block}
    
    \framebreak
    
    \begin{block}{Do not be shy, ask!}
        %
        \begin{itemize}
            %
            \item you are not alone!
            %
            \item interact with your colleagues
            %
            \item if you have questions / solutions about problems that could interest others share them on the \bluehref{https://virtuale.unibo.it/mod/forum/view.php?id=1284514}{general forum}
            %
            \item if you have problems regarding your project (e.g., technical issues, design doubts, etc.) you can write us an email or take an appointment
            %
        \end{itemize}
        %        
    \end{block}
    %
\end{frame}

%===============================================================================
\section*{}
%===============================================================================
\frame{\titlepage}

%===============================================================================
\section*{\bibname}
%===============================================================================

\setbeamertemplate{page number in head/foot}{}
%\\\\\\\\\\\\\\\\\\\\\
\begin{frame}[t,allowframebreaks,noframenumbering]\frametitle{\refname}
    % \begin{frame}[c]\frametitle{\refname}
    %    \footnotesize
    %    \scriptsize
        \tiny
    \bibliographystyle{plain}
    \bibliography{sd-lab-about}
\end{frame}
%\\\\\\\\\\\\\\\\\\\\\

%%%%%%%%%%%%%%%%%%%%%%%%%%%%%%%%%%%%%%%%%%%%%%%%%%%%%%%%%%%%%%%%%%%%%%%%%%%%%%%
\end{document}
%%%%%%%%%%%%%%%%%%%%%%%%%%%%%%%%%%%%%%%%%%%%%%%%%%%%%%%%%%%%%%%%%%%%%%%%%%%%%%%%
